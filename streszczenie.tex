Nowe, coraz dokładniejsze pomiary właściwości istotnych biologicznie cząsteczek podważają nasze rozumienie makromolekuł jako cząsteczek o dobrze określonych, stałych kształtach określających ich funkcję.
Rozkład równowagowy ich konformacji opisuje rozkład Boltzmanna, który łączy w sobie następujące dwa człony: wpływ temperatury rozpuszczalnika oraz energię potencjalną danej konfiguracji, która z kolei opisuje właściwości elastyczne danej cząsteczki.
Zachowanie elastycznych makromolekuł jest zatem kształtowane przez konkurencję tych dwóch zjawisk fizycznych -- gdy studnie potencjału są płytkie w porównaniu z typowymi fluktuacjami termicznymi, występuje szeroka gama konformacji; i odwrotnie, gdy są głębokie, można zaobserwować jedynie niewielkie odchylenia od konfiguracji minimalizujących energię.

W niniejszej rozprawie doktorskiej przedstawiono teoretyczny opis zmienności konformacyjnej elastycznych makromolekuł i jej wpływu na ich dyfuzję.
Pierwsza część rozprawy zawiera przegląd podstaw teoretycznych wymaganych do budowy modeli gruboziarnistych, które stanowią istotę rozprawy oraz teoretyczne podstawy metod eksperymentalnych stosowanych do weryfikacji założeń upraszczających owych modeli.
Druga część rozprawy składa się z szeregu powiązanych tematycznie publikacji i preprintów, w których prezentujemy metody modelowania molekuł w szerokim spektrum sprężystości.

Zaczynając od cząsteczek o dalekim zasięgu korelacji elastycznych (ang. persistence length) w porównaniu z ich rozmiarem, pokazujemy jak modelować zbliżanie bardzo krótkiego segmentu DNA do nanoporu i analizujemy wpływ ścian i anizotropii hydrodynamicznej w procesie wychwytywania cząstek przez nanopory.
Ustalamy teoretyczne kryteria tego, w jakich przypadkach konieczne jest uwzględnienie oddziaływania cząsteczek ze ściankami, przybliżając molekuły przez pręt z równomiernie rozłożonym ładunkiem elektrycznym.
Ponadto, badamy wpływ ujemnego trzeciorzędowego skręcenia (ang. supercoiling) i krzywizny DNA na właściwości hydrodynamiczne mini-pętli DNA o długości 336 bp i 672 bp.
Wykorzystując liniową teorię elastyczności i modelowanie hydrodynamiczne, przewidujemy kształty DNA i ich współczynniki dyfuzji.
Dane eksperymentalne współczynników dyfuzji i sedymentacji uzyskane za pomocą ultrawirowania analitycznego pokazują dobrą zgodność z naszymi przewidywaniami teoretycznymi.

Dla pośrednich wartości zasięgu korelacji elastycznych wyznaczamy taki zakres długości i sił zewnętrznych, przy którym sedymentacja elastycznej, cienkiej pętli pozostaje stabilna na wyboczenia.
Nasza analiza, oparta na liniowej teorii elastyczności połączonej z lokalnym oporem hydrodynamicznym, wyznacza kryterium stabilności zależne od pojedynczego bezwymiarowego parametru.

W bardziej ogólnym przypadku, gdy istotne są zarówno fluktuacje termiczne, jak i siły sprężyste, prezentujemy podejście numeryczne.
Nasze podejście opiera się na algorytmie całkowania stochastycznych równań różniczkowych połączonym z oddziaływaniami hydrodynamicznymi opartymi na przybliżeniu Rotne-Pragera.
Powyższe metody zostały udostępnione jako zbiór paczek w Pythonie zaprojektowanych w celu szybkiego programowania i przeliczania małych symulacji dynamiki Browna.

W przypadku białek nieustrukturyzowanych (ang. IDP), które reprezentują przeciwną skrajność elastyczności w porównaniu z bardzo sztywnym DNA, objętość wykluczona jest determinantą dystrybucji w rozkładach równowagowych.
Przedstawiamy model konformacji Globule-Linker, który w połączeniu z Przybliżeniem Minimalnej Dyssypacji pozwala obliczyć wielkość hydrodynamiczną takich białek.
Porównując podejście gruboziarniste z największym jak dotąd zmierzonym zbiorem danych eksperymentalnych, pokazujemy, że nasze podejście umocowane w prawach fizycznych jest skuteczniejsze niż dopasowania fenomenologiczne dostępne w literaturze.

W ostatnim artykule rozważamy teoretyczny problem rozkładów równowagowych cząsteczek o wielu stopniach swobody z których część jest silnie związana (jak na przykład długości wiązań chemicznych).
Pokazujemy, że istotne detale potencjałów realizujących więzy zostały przeoczone we wcześniejszych pracach oraz proponujemy właściwą metodę ich uwzględnienia.

Nasze wyniki, obejmujące szeroki zakres korelacji elastycznych, pokazują różnorodność zjawisk wynikających z współwystępowania elastyczności i fluktuacji termicznych.
Aby ułatwić analizę takich układów, stworzyliśmy zbiór otwartych narzędzi numerycznych, które zostały opublikowane wraz z dokumentacją.
Możliwość zastosowania powyższych narzędzi przetestowano bezpośrednio w stosunkowo sztywnym biopolimerze (DNA) i w miękkich strukturach (IDP).
Przedstawione narzędzia zapewniają wgląd w pośrednie reżimy sztywności, w których fluktuacje termiczne konkurują z interakcjami wewnątrzcząsteczkowymi i mogą posłużyć do lepszego zrozumienia zjawisk elastohydrodynamicznych w mikroskali.
