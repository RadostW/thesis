\grayout{Pierwsza publikacja pt. \emph{Stabilność sedymentujących elastycznych pętli}
    współautorami Piotra Szymczaka i Macieja Lisickiego
    zapewnia liniową analizę stabilności pętli sprężystych w ramach teorii siły oporu w połączeniu z siłami sprężystymi modelowanymi za pomocą równania Eulera-Bernoulliego.
    Udało nam się ustalić półanalityczne kryterium stabilności i ponownie wyprowadzić bezwymiarową wielkość regulującą niestabilność wyboczeniową dla tego i podobnych problemów.

    Druga publikacja pt.
    \emph{Efekty hydrodynamiczne w wychwytywaniu pręcików przez nanopor}
    współautor: Maciej Lisicki
    przedstawia analizę wpływu oddziaływania ścianek i anizotropii hydrodynamicznej na proces wychwytywania nanoporów.
    Teoretyczne rozważenie cząsteczki przypominającej pręcik z równomiernie rozłożonym ładunkiem dostarcza prostych, opartych na skalowaniu kryteriów pozwalających określić, kiedy i gdzie wymagane jest uwzględnienie poprawek ścian.

    Trzecia publikacja zatytułowana \emph{Pychastic: Precise Brownian dynamics using integrators Taylor-Ito w Pythonie} współautorami: Maciej Bartczak, Kamil Kolasa i Maciej Lisicki jest wynikiem prac nad implementacją wydajnych rozwiązań stochastycznych równań różniczkowych, zdolnych do wygodnego rozwiązywania problemów dynamiki Browna (BD).
    Wyrażając równania BD jako całki Itō, możemy wykorzystać klasyczne metody obciętych integratorów Taylora-Itō.
    W ramach dokumentacji pakietu \code{pychastic} pokazujemy, jak radzić sobie z typowymi przeszkodami BD: obliczeniami rozbieżności tensora ruchliwości w równaniu dyfuzji oraz nieciągłymi trajektoriami spotykanymi podczas pracy z dynamiką na $S^2$ i $SO(3)$.
    Dzięki implementacji zorientowanej na wektoryzację osiągnęliśmy wydajność porównywalną z wcześniejszymi implementacjami w językach programowania niższego poziomu.

    Czwarta publikacja zatytułowana \emph{Kształty wywołane superskręceniem DNA zmieniają właściwości hydrodynamiczne minikola} współautorami: Maduni Ranasinghe, Jonathan M Fogg, Daniel J Catanese Jr, Maria L Ekiel-Jeżewska, Maciej Lisicki, Borries Demeler, Lynn Zechiedrich i Piotr Szymczak jest efektem teoretyczno-eksperymentalnej współpracy z zespołem z Baylor College of Medicine (odpowiedzialnym za biosyntezę) oraz zespołem z University of Lethbridge (odpowiedzialnym za analityczne eksperymenty ultrawirowania).
    W tej publikacji określiliśmy wpływ ujemnego superskręcenia i krzywizny na właściwości hydrodynamiczne DNA, poddając minikola DNA o wielkości 336 bp i 672 bp analitycznemu ultrawirowaniu (AUC).
    Następnie wykorzystaliśmy teorię sprężystości liniowej i obliczenia hydrodynamiczne, aby przewidzieć kształty DNA i współczynniki dyfuzji.

    Piąty manuskrypt zatytułowany \emph{Przybliżenie minimalnego rozproszenia: szybki algorytm przewidywania właściwości dyfuzyjnych białek wewnętrznie nieuporządkowanych} współautorzy: Agnieszka Michaś, Michał K.
    Białobrzewski, Barbara Klepka, Maja Cieplak-Rotowska, Zuzanna Staszałek, Bogdan Cichocki, Maciej Lisicki, Piotr Szymczak i Anna Niedźwiecka jest efektem współpracy eksperymentalno-teoretycznej z zespołem Instytutu Fizyki PAN (odpowiedzialnym za biosyntezę i spektroskopię korelacyjną fluorescencji).
    W naszym badaniu demonstrujemy szybką metodę numeryczną łączącą proste próbkowanie konformacyjne i przybliżone interakcje hydrodynamiczne w celu oszacowania współczynników dyfuzji białek wewnętrznie nieuporządkowanych (IDP), nawet w obecności domen strukturalnych, z precyzją przewyższającą klasyczne przybliżenie Kirkwooda.
    Dzięki nowemu zbiorowi pomiarów współczynników dyfuzji możemy ilościowo porównać nasze przewidywania z wieloma modelami obecnymi w literaturze (takimi jak prawa potęgowe i prawa potęgowe z poprawkami zależnymi od sekwencji).

    Szósty rękopis zatytułowany \emph{Ponowne spojrzenie na paradoks trimeru} współautorem Maciej Lisicki zajmuje się problemem bardzo sztywnych wiązań i wynikających z nich ograniczających rozkładów kątów wiązania.
    Pokazuje, że chociaż rozwiązanie paradoksu zostało wyjaśnione już w 1984 roku w \textcite{van_Kampen_1984}, w dobrze znanych książkach, takich jak \textcite{Frenkel_2002}, brakowało wyraźnego (i prawidłowego) potraktowania tej granicy.
    Poprzez dokładniejsze potraktowanie rozkładów osobliwych pokazujemy, że do prawidłowego określenia kątów wiązań wymagana jest kombinacja właściwości metrycznych rozmaitości ograniczającej i hesjanu pola ograniczającego, a rozkłady „równomierne na kuli” dla harmonicznych potencjałów ograniczających nie są uniwersalne , z potencjalnie dużymi odchyleniami dla małych cząsteczek cyklicznych.
    Wyniki te ustanawiają teoretyczne podstawy modelu globula-łącznik i powinny wyznaczać kierunki dalszych prac nad przybliżeniem minimalnego rozproszenia.

    Cel prezentowanej pracy był dwojaki: zaspokojenie bezpośrednich potrzeb grup eksperymentalnych, z którymi współpracowaliśmy, ale także ustalenie solidnych modeli „hipotezy zerowej”, które są wystarczająco łatwe w użyciu i uwzględniają wszystkie podstawowe interakcje wymagane do modelowania dyfuzyjne, ale żadnych więcej interakcji.
    Dzięki temu odchylenia od tych modeli można wykorzystać jako ilościowe wskaźniki istotnego udziału nowych zjawisk fizycznych (na przykład oddziaływań elektrostatycznych lub powstawania przejściowych mostków pomiędzy odległymi częściami cząsteczki).
}