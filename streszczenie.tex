Nowe, coraz dokładniejsze pomiary właściwości istotnych biologicznie cząsteczek powodują napięcie w naszym rozumieniu makromolekuł jako cząstek posiadających dobrze określone, stałe kształty, które określają ich funkcję.
Rozkład równowagowy ich konformacji opisuje rozkład Boltzmanna, który łączy w sobie dwa człony: wpływ temperatury rozpuszczalnika oraz energię potencjalną danej konfiguracji, która z kolei oddaje właściwości elastyczne danej cząsteczki.
Zachowanie elastycznych makromolekuł zależy od współzawodnictwa dwóch zjawisk fizycznych - ilekroć studnie potencjału są płytkie w porównaniu z typowymi fluktuacjami termicznymi, występuje szeroka gama konformacji; i odwrotnie, gdy są głębokie, można zaobserwować jedynie niewielkie odchylenia od konfiguracji minimalizujących energię.

W rozprawie doktorskiej przedstawiono teoretyczny opis zmienności konformacyjnej elastycznych makromolekuł i jej wpływu na dyfuzję.
Pierwsza część rozprawy zawiera przegląd podstaw teoretycznych wymaganych do budowy modeli gruboziarnistych, które stanowią rdzeń tej rozprawy oraz teoretyczne podstawy metod eksperymentalnych stosowanych do sprawdzania założeń upraszczających przyjętych takich modeli.

Druga część rozprawy składa się z szeregu powiązanych tematycznie publikacji i preprintów, w których prezentujemy metody modelowania molekuł w szerokim spektrum sprężystości.

Zaczynając od cząsteczek o dalekim zasięgu korelacji elastycznych (ang.
persistence length) w porównaniu z ich rozmiarem, pokazujemy, jak modelować zbliżanie bardzo krótkiego segmentu DNA do nanoporu i analizujemy wpływ ścian i anizotropii hydrodynamicznej w procesie wychwytywania cząstek przez nanopory.
Ustalamy teoretyczne kryteria, kiedy i gdzie konieczne jest uwzględnienie oddziałwyania ze ściankami, przybliżając cząsteczkę przez pręt z równomiernie rozłożonym ładunkiem.
Po drugie, badamy wpływ ujemnego trzeciorzędowego skręcenia (ang.
supercoiling) i krzywizny na właściwości hydrodynamiczne minipętli DNA o długości 336 bp i 672 bp.
Wykorzystując liniową teorię elastyczności i modelowanie hydrodynamiczne, przewidujemy kształty DNA i ich współczynniki dyfuzji.
Dane eksperymentalne współczynników dyfuzji i sedymentacji uzyskanymi za pomocą ultrawirowania analitycznego pokazują dobrą zgodność z naszymi przewidywaniami teoretycznymi.

Dla pośrednich wartości zasięgu korelacji elastycznych wyznaczamy zakres długości i sił zewnętrznych, przy których sedymentacja elastycznej, cienkiej pętli pozostaje stabilna na wyboczenia.
Nasza analiza, oparta na liniowej teorii elastyczności połączonej z lokalnym oporem hydrodynamicznym daje kryterium stabilności zależne od pojedynczego bezwymiarowego parametru.

W bardziej ogólnym przypadku, gdy istotne są zarówno fluktuacje termiczne, jak i siły sprężyste, prezentujemy zestaw metod numerycznych.
Nasze podejście opiera się na algorytmie całkowania stochastycznych równań różniczkowych połączonym z oddziaływaniami hydrodynamicznymi opartymi na przybliżeniu Rotne-Pragera.
Te metody zostały zawarte w zestawie pakietów do Pythona zaprojektowanych w celu szybkiego tworzenia i symulowania małych symulacji dynamiki Browna dzięki akceleracji sprzętowej.

Wykazujemy, że w przypadku białek nieustrukturyzowanych (IDP), które reprezentują przeciwną skrajność elastyczności w porównaniu z bardzo sztywnym DNA, to objętość wykluczona jest determinantą dystrybucji w rozkładach równowachowych.
Przedstawiamy model konformacji Globule-Linker który w połączeniu z Przybliżeniem Minimalnej Dyssypacji pozwala obliczyć wielkość hydrodynamiczną takich białek.
Porównując podejście gruboziarniste z największym jak dotąd zestawem wartości eksperymentalnych, pokazujemy, że nasze fizycznie umocowane podejście jest skuteczniejsze niż dopasowania fenomenologiczne dostępne w literaturze.

W ostatnim manuskrypcie rozważamy teoretyczny problemem rozkładów równowagowych cząsteczek o wielu stopniach z których część jest silnie związana (takich jak na przykład długości wiązań chemicznych).
Pokazujemy, że istotne detale potencjałów realizujących wiązy zostały przeoczone we wcześniejszych pracach oraz właściwą metodę ich uwzględnienia.

Nasze wyniki, obejmujące szeroki zakres zasięgu korelacji elastycznych, pokazują bogatą różnorodność zjawisk wynikających z współzawodnictwa elastyczności i fluktuacji termicznych.
Aby ułatwić analizę takich układów, stworzyliśmy szereg otwartych narzędzi numerycznych, które zostały opublikowane wraz z dokumentacją.
Możliwość zastosowania tych narzędzi przetestowano bezpośrednio w stosunkowo sztywnym biopolimerze -- DNA -- i w miękkich strukturach IDP.
Przedstawione narzędzia zapewniają wgląd w pośrednie reżimy sztywności, w których fluktuacje termiczne konkurują z interakcjami wewnątrzcząsteczkowymi, i można je wykorzystać do lepszego zrozumienia zjawisk elastohydrodynamicznych w mikroskali.
