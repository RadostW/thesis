Nowe, coraz dokładniejsze pomiary właściwości istotnych biologicznie cząsteczek podważają nasze rozumienie makromolekuł jako cząsteczek o dobrze określonych, stałych kształtach określających ich funkcję.
Rozkład równowagowy ich konformacji opisuje rozkład Boltzmanna, który łączy dwa efekty: wpływ temperatury rozpuszczalnika oraz energię potencjalną danej konfiguracji, która z kolei opisuje właściwości elastyczne cząsteczki.
Zachowanie elastycznych makromolekuł jest zatem określone przez współzawodnictwo dwóch zjawisk fizycznych: gdy studnie potencjału są płytkie w porównaniu z typowymi fluktuacjami termicznymi, występuje szeroka gama konformacji; z drugiej strony, gdy są one głębokie, można zaobserwować jedynie niewielkie odchylenia od konfiguracji minimalizujących energię makromolekuły.

W niniejszej rozprawie doktorskiej przedstawiono teoretyczny opis zmienności konformacyjnej elastycznych makromolekuł i jej wpływu na ich dyfuzję.
Pierwsza część rozprawy zawiera przegląd podstaw teoretycznych wymaganych do budowy modeli gruboziarnistych, które stanowią istotę rozprawy oraz teoretyczne podstawy metod eksperymentalnych stosowanych do weryfikacji upraszczających założeń, wykorzystanych do konstrukcji modeli.
Druga część rozprawy składa się z szeregu powiązanych tematycznie publikacji i preprintów, w których prezentujemy metody modelowania makromolekuł w szerokim zakresie ich sprężystości.

Zaczynając od cząsteczek o dalekim zasięgu korelacji elastycznych (tzn. o dużej długości persystencji, ang. persistence length) w porównaniu z ich rozmiarem, pokazujemy jak modelować zbliżanie się krótkiego fragmentu DNA do nanoporu. Analizujemy przy tym wpływ ścianek i anizotropii hydrodynamicznej makrocząstki w procesie wychwytywania cząstek przez nanopory.
Określamy teoretyczne kryteria dla przypadków, w jakich konieczne jest uwzględnienie oddziaływania hydrodynamicznego cząsteczek ze ściankami, modelując kształt cząstki jako pręt z równomiernie rozłożonym ładunkiem elektrycznym.
W kolejnych pracach badamy wpływ ujemnego trzeciorzędowego skręcenia (ang. supercoiling) i krzywizny DNA na właściwości hydrodynamiczne minipętli DNA o długości 336 i 672 par zasad.
Wykorzystując liniową teorię elastyczności i modele hydrodynamiczne, przewidujemy kształty DNA i ich współczynniki dyfuzji w roztworze w zależności od stopnia skręcenia. 
Pomiary eksperymentalne współczynników dyfuzji i sedymentacji uzyskane za pomocą ultrawirowania analitycznego wykazują dobrą zgodność z naszymi przewidywaniami teoretycznymi. Dla pośrednich wartości zasięgu korelacji elastycznych wyznaczamy zakres długości i skalę sił zewnętrznych, przy którym sedymentacja elastycznej, cienkiej pętli pozostaje stabilna na wyboczenia.
Nasza analiza, oparta na liniowej teorii elastyczności połączonej z teorią lokalnego oporu hydrodynamicznego, wyznacza kryterium stabilności zależne od pojedynczego bezwymiarowego parametru. W bardziej ogólnym przypadku, gdy istotne są zarówno fluktuacje termiczne, jak i siły sprężyste, prezentujemy podejście numeryczne. Opiera się ono na algorytmie całkowania stochastycznych równań różniczkowych, połączonym z modelem oddziaływań hydrodynamicznych opartym na przybliżeniu Rotne-Pragera. Powyższe metody zostały udostępnione jako zbiór paczek w Pythonie zaprojektowanych w celu szybkiego programowania i wykonywania symulacji dynamiki Brownowskiej.

W przypadku białek nieustrukturyzowanych (ang. intrinsically disordered proteins, IDP), które reprezentują przeciwny biegun spektrum elastyczności w porównaniu z bardzo sztywnym DNA i mogą być rozpatrywane jako bardzo wiotkie, determinantą ich konformacji w rozkładach równowagowych jest wykluczona objętość. 
Do modelowania białek IDP, proponujemy model konformacji GLM (ang. globule-linker model), który w połączeniu z hydrodynamicznym przybliżeniem minimalnej dyssypacji pozwala obliczyć wielkość hydrodynamiczną takich białek.
Porównując nasze gruboziarniste podejście teoretyczne z największym jak dotąd zmierzonym zbiorem danych eksperymentalnych pokazujemy, że nasze podejście jest skuteczniejsze niż dopasowania fenomenologiczne dostępne w literaturze. W ostatnim artykule rozważamy teoretyczny problem rozkładów równowagowych cząsteczek o wielu stopniach swobody, z których część jest silnie związana (jak na przykład długości wiązań chemicznych).
Pokazujemy, że istotne detale potencjałów realizujących więzy zostały przeoczone we wcześniejszych pracach oraz proponujemy właściwą, ścisłą matematycznie metodę ich uwzględnienia.

Nasze wyniki, obejmujące szeroki zakres korelacji elastycznych, pokazują różnorodność zjawisk wynikających z współwystępowania elastyczności, sił lepkich i fluktuacji termicznych.
Aby ułatwić analizę takich układów, stworzyliśmy zbiór otwartych narzędzi numerycznych, które zostały opublikowane wraz z dokumentacją.
Możliwość zastosowania powyższych narzędzi przetestowaliśmy bezpośrednio w stosunkowo sztywnym biopolimerze (minikółka z DNA) i w miękkich strukturach białek nieuporządkowanych (IDP).
Przedstawione narzędzia dają wgląd w pośrednie reżimy sztywności, w których fluktuacje termiczne konkurują z oddziaływaniami wewnątrzcząsteczkowymi. Liczymy, że proponowane modele mogą posłużyć do lepszego zrozumienia zjawisk elastohydrodynamicznych w mikroskali.
